\documentclass[12pt]{article}
\usepackage[utf8]{inputenc}
\usepackage[spanish]{babel}
\usepackage{geometry}
\usepackage{graphicx}
\usepackage{tikz}
\usepackage{xcolor}
\usepackage{setspace}
\usepackage{url}
\usepackage{hyperref}
\usepackage{titlesec}
\usepackage{fancyhdr}
\usepackage{lastpage}
\usepackage{amsmath} % Para fórmulas matemáticas
\geometry{
    a4paper,
    top=3cm,
    bottom=3cm,
    left=2.5cm,
    right=2.5cm
}

\renewcommand{\baselinestretch}{1.2}

% Hipervínculos
\hypersetup{
    colorlinks,
    linkcolor=black,
    citecolor=black,
    urlcolor=black
}

% Encabezados y pies
\fancyhf{}
\pagestyle{fancy}
\fancyhead[L]{\textcolor{black}{F1 Garage Manager}}
\fancyhead[R]{\textcolor{black}{CE-3101 Bases de Datos}}
\fancyfoot[C]{\fontsize{10}{12}\selectfont Página \thepage\ de \pageref{LastPage}}
\renewcommand{\headrulewidth}{0.5pt}
\renewcommand{\footrulewidth}{0pt}

% Comando para la portada
\newcommand{\portada}{
    \begin{titlepage}
        \thispagestyle{empty}
        \pagecolor{white}
        \color{black}
        
        \vspace*{0.8cm}
        
        {\centering
        \begin{minipage}[c]{0.3\textwidth}
            \centering
            \includegraphics[width=1\textwidth]{logo-tec.png} % Asegúrate de subir esta imagen
        \end{minipage}
        }
        
        \vspace{1.2cm}
        {\noindent\rule{\linewidth}{1.5pt}}
        \vspace{.3cm}
        
        {\centering
        \fontsize{28}{34}\selectfont
        \textbf{F1 Garage Manager}\\
        \vspace{0.5cm}
        }
        
        {\centering
        \fontsize{12}{14}\selectfont
        \textit{Sistema integral de gestión de equipos y simulación de carreras de Fórmula 1}\\
        }
        
        \vspace{1cm}
        {\noindent\rule{\linewidth}{1.5pt}}
        
        \vfill
        
        {\centering
        \fontsize{14}{17}\selectfont
        \textbf{Entregable \#1: Modelo Conceptual}\\
        }
        
        \vspace{1.1cm}
        {\centering
        \fontsize{10}{12}\selectfont
        Trabajo presentado para la obtención de evaluación en\\
        }
        
        \vspace{0.3cm}
        {\centering
        \fontsize{12}{14}\selectfont
        \textbf{CE-3101 Bases de Datos}\\
        }
        
        \vspace{0.8cm}
        {\centering
        \fontsize{11}{13}\selectfont
        \textit{Modalidad Verano Intensivo}\\
        }
        
        \vspace{0.2cm}
        {\centering
        \fontsize{10}{12}\selectfont
        Diciembre 2025 – Enero 2026\\
        }
        
        \vspace{0.8cm}
        {\centering
        \fontsize{9}{11}\selectfont
        Instituto Tecnológico de Costa Rica\\
        \vspace{0.05cm}
        Escuela de Ingeniería en Computadores\\
        }
        
        \vfill
        
        {\centering
        \fontsize{10}{12}\selectfont
        \begin{tabular}{ll}
        \textbf{Integrantes} & Steven Aguilar Alvarez | 2024202865\\
                             & Sebastián Chaves Ruiz | 2021032506\\
                             & Ian Yoel Gómez Oses | 2023216136\\
                             & Mauro Brenes Brenes | 2023213314\\
        \\
        \textbf{Profesor}    & MSc. Andrés Vargas Rivera \\
        \end{tabular}
        }
        
        \vfill
        {\centering
        \fontsize{9}{11}\selectfont
        \today\\
        }
        \vspace{0.8cm}
    \end{titlepage}
    \pagecolor{white}
    \color{black}
}

\begin{document}

\portada

\tableofcontents
\newpage

\section{Introducción}
Este proyecto corresponde al desarrollo de un sistema de información para la gestión de equipos de Fórmula 1, denominado \textbf{F1 Garage Manager}. El objetivo primordial es aplicar conceptos de modelado de bases de datos relacionales para administrar de forma íntegra equipos, conductores, componentes de vehículos y la simulación de carreras basada en parámetros de rendimiento físico y técnico. El dominio abarca desde la gestión financiera mediante patrocinios hasta la ingeniería detallada de configuración de monoplazas.

\section{Diagrama Entidad-Relación (ER)}
En esta sección se presenta el modelo conceptual que representa las entidades del negocio y sus interacciones.

\subsection{Simbología utilizada}
De acuerdo con los requerimientos del curso, se ha utilizado la siguiente nomenclatura visual:
\begin{itemize}
    \item \textbf{Rectángulo Azul:} Entidades principales.
    \item \textbf{Rombo Blanco:} Relaciones entre entidades.
    \item \textbf{Óvalo Amarillo (Subrayado):} Atributos clave (Primary Keys).
    \item \textbf{Óvalo Verde:} Atributos simples.
    \item \textbf{Óvalo Gris (Punteado):} Atributos derivados.
\end{itemize}

\subsection{Descripción de Entidades}
El modelo integra 13 entidades principales organizadas en cuatro dominios funcionales:

\subsubsection{Dominio: Gestión Organizacional}

\begin{itemize}
    \item \textbf{Equipo:} Almacena información del equipo y su presupuesto derivado (calculado a partir de los Aportes de Patrocinadores). Restricción: máximo 2 Carros por Equipo.
    
    \item \textbf{Patrocinador:} Entidad que financia a los Equipos. Almacena identificador único, nombre y correo electrónico para contacto.
    
    \item \textbf{Aporte:} Registra contribuciones monetarias de Patrocinadores a Equipos. Contiene fecha del aporte, monto y descripción del patrocinio.
    
    \item \textbf{Usuario:} Gestiona el acceso al sistema con roles definidos (Admin, Engineer, Driver). Almacena identificador, correo, hash de contraseña. Un Usuario opera exactamente 1 Equipo.
\end{itemize}

\subsubsection{Dominio: Recursos Humanos}

\begin{itemize}
    \item \textbf{Conductor:} Piloto asignado a un Equipo. Registra identificador único, nombre y habilidad ($H$) en rango $[0, 100]$ (entero). La habilidad afecta directamente al rendimiento en simulaciones.
\end{itemize}

\subsubsection{Dominio: Componentes y Configuración}

\begin{itemize}
    \item \textbf{Pieza (Part):} Catálogo de componentes con identificador único, nombre, precio y tres atributos de rendimiento (enteros $[0, 9]$): potencia ($p$), aerodinámica ($a$) y manejo ($m$).
    
    \item \textbf{PartCategory:} Categoría de piezas (exactamente 5 obligatorias: Unidad de potencia, Paquete aerodinámico, Neumáticos, Suspensión, Caja de cambios). Almacena identificador único y nombre.
    
    \item \textbf{PartStock:} Registro de disponibilidad en la tienda. Almacena identificador de Pieza, cantidad disponible en stock y fecha de última actualización.
    
    
    \item \textbf{Carro:} Monoplaza del Equipo. Representa el vehículo el cual debe poseer exactamente 5 categorías de piezas instaladas para considerarse ``Finalizado''. Máximo 2 por Equipo.
    
    \item \textbf{CarSetup:} Configuración específica de un Carro en un momento dado. Almacena identificador único, fecha de instalación y tres atributos derivados: $P_{total}$, $A_{total}$, $M_{total}$ (suma de valores de rendimiento de las 5 Piezas).
\end{itemize}

\subsubsection{Dominio: Circuitos y Simulaciones}

\begin{itemize}
    \item \textbf{Circuito:} Pista de carrera. Almacena identificador único, nombre, distancia total ($D$ en kilómetros) y cantidad de curvas ($C$).
    
    \item \textbf{Simulación:} Registro de eventos de carrera en circuitos específicos. Almacena identificador único, fecha/hora de ejecución e identificador del Circuito. Una Simulación genera múltiples ResultadoSimulación.
    
    \item \textbf{ResultadoSimulación:} Resultado individual de un Carro en una Simulación. Almacena el identificador del Carro, configuración utilizada, velocidades calculadas ($V_{recta}$, $V_{curva}$), penalización, tiempo total en segundos y posición final (ranking).
\end{itemize}

\subsection{Diagrama ER}
\begin{figure}[h]
    \centering
    % Reemplaza 'diagrama_er.png' con el nombre de tu archivo subido
    \includegraphics[width=1\linewidth]{er.png} 
    \caption{Modelo Conceptual ER - F1 Garage Manager}
\end{figure}

    \section{Reglas de Negocio}
A continuación, se describen las restricciones y lógicas aplicadas al modelo:

\subsection{Gestión Financiera y Compras}
\begin{itemize}
    \item El presupuesto de un equipo se calcula exclusivamente a partir de los aportes registrados de sus patrocinadores.
    \item Para realizar una compra, el sistema valida que exista stock disponible en la tienda y que el equipo posea presupuesto suficiente.

\end{itemize}

\subsection{Armado y Configuración del Carro}
\begin{itemize}
    \item Un equipo puede poseer un máximo de \textbf{2 carros}.
    \item Para que un carro se considere "Finalizado", debe tener instalada exactamente una pieza de cada una de las 5 categorías: Unidad de potencia, Paquete aerodinámico, Neumáticos, Suspensión y Caja de cambios.
\end{itemize}

\subsection{Lógica de Simulación}
La simulación calcula el rendimiento basándose en las siguientes fórmulas matemáticas:

\textbf{Velocidades ($km/h$):}
\begin{itemize}
    \item $V_{recta} = 200 + 3P + 0.2H - A$
    \item $V_{curva} = 90 + 2A + 2M + 0.2H$
\end{itemize}

\textbf{Tiempo y Penalización:}
\begin{itemize}
    \item $Penalizacion (seg) = \frac{C \cdot 40}{1 + (H/100)}$
    \item $Tiempo_{total} (seg) = (Tiempo_{horas} \cdot 3600) + Penalizacion$
\end{itemize}
Donde $P, A, M$ son los totales del carro, $H$ la habilidad del conductor y $C$ la cantidad de curvas del circuito.

\newpage

\section{Análisis de Cardinalidad}

Las siguientes son las 15 relaciones principales del modelo con sus respectivas cardinalidades:

\begin{center}
\begin{tabular}{|l|c|p{5cm}|}
\hline
\textbf{Relación} & \textbf{Card.} & \textbf{Descripción} \\
\hline
Patrocinador \textit{realiza} Aporte & $(1, N)$ & Un patrocinador realiza muchos aportes \\
\hline
Aporte \textit{recibe} Equipo & $(N, 1)$ & Muchos aportes van a un equipo \\
\hline
Usuario \textit{opera} Equipo & $(1, 1)$ & Un usuario opera un equipo (Engineer) \\
\hline
Equipo \textit{tiene} Carro & $(1, N)$ & Un equipo tiene hasta 2 carros máximo \\
\hline
Conductor \textit{pertenece\_a} Equipo & $(N, 1)$ & Muchos conductores pertenecen a un equipo \\
\hline
Equipo \textit{posee} Pieza & $(1, N)$ & Un equipo contiene muchas piezas \\
\hline
PartCategory \textit{contiene} Pieza & $(1, N)$ & Una categoría contiene muchas piezas \\
\hline
CarSetup \textit{instala} Pieza & $(N, M)$ & Un setup instala 5 piezas (una por categoría) \\
\hline
Carro \textit{configurado\_por} CarSetup & $(1, N)$ & Un carro tiene múltiples configuraciones \\
\hline
Circuito \textit{se\_ejecuta\_en} Simulación & $(1, N)$ & Un circuito alberga muchas simulaciones \\
\hline
Simulación \textit{genera} ResultadoSimulación & $(1, N)$ & Una simulación genera un resultado por carro \\
\hline
Usuario \textit{ejecuta} Simulación & $(1, N)$ & Un Admin ejecuta muchas simulaciones \\
\hline
Carro \textit{genera} ResultadoSimulación & $(1, N)$ & Un carro genera muchos resultados en diferentes simulaciones \\
\hline
CarSetup \textit{para\_categoria} PartCategory & $(N, 1)$ & Muchos setups pueden usar piezas de una categoría \\
\hline
Pieza \textit{tiene\_stock} PartStock & $(1, 1)$ & Una pieza tiene exactamente un registro de stock \\
\hline
\end{tabular}
\end{center}

\newpage

\section{Referencias}

Elmasri, R., \& Navathe, S. B. (2016). \textit{Fundamentals of database systems} (7th ed.). Addison-Wesley. \\

Instituto Tecnológico de Costa Rica. (2025). \textit{Especificación del Proyecto: F1 Garage Manager}. CE-3101 Bases de Datos.

\end{document}